% Created 2015-07-26 Sun 05:26
\documentclass[11pt]{article}
\usepackage[utf8]{inputenc}
\usepackage[T1]{fontenc}
\usepackage{fixltx2e}
\usepackage{graphicx}
\usepackage{longtable}
\usepackage{float}
\usepackage{wrapfig}
\usepackage{rotating}
\usepackage[normalem]{ulem}
\usepackage{amsmath}
\usepackage{textcomp}
\usepackage{marvosym}
\usepackage{wasysym}
\usepackage{amssymb}
\usepackage{hyperref}
\tolerance=1000
\usepackage{minted}
\author{Ben Mezger}
\date{\textit{<2015-07-25 Sat>}}
\title{Insertion sort}
\hypersetup{
  pdfkeywords={},
  pdfsubject={},
  pdfcreator={Emacs 24.5.1 (Org mode 8.2.10)}}
\begin{document}

\maketitle

\section{Introduction}
\label{sec-1}
We need to sort numbers quite a lot when programming, for example, suppose you
wrote a computer card game, each card will be shuffled at the beginning, and
each user starts with 6 cards, but they need to be sorted, like the following.

\begin{table}[htb]
\caption{\label{fig:-ss-cards}Shuffled card vs sorted cards.}
\centering
\begin{tabular}{lrrrrrr}
Shuffled & 6 & 3 & 1 & 4 & 2 & 5\\
Sorted & 1 & 2 & 3 & 4 & 5 & 6\\
\end{tabular}
\end{table}

\section{Solution}
\label{sec-2}

One solution for this problem, is to use an algorithm called \emph{insertion
sort}. Insertion sort is fairly simple, but it's a commonly used algorithm. Insertion
sort works like we human sort cards. say we have 6 cards lying in the table,
the cards are from 1-6, but they are not sorted. the cards on the table looks
like the shuffled line from \texttt{\emph{Table 1}}. Our algorithm, needs to take care of two
elements, the first element (\emph{i}) and the second element, (\emph{j}). While \emph{j}
points to the second element, \emph{i} points to \emph{j - 1}. We do so, because later, we
can compare \emph{j} with \emph{i} and check if \emph{i} $\ge$ \emph{j}, if so, we swap the element
\emph{i} with the \emph{j} each time, until \emph{i} $\le$ \emph{j}.

\begin{center}
\begin{tabular}{lrrrrrr}
\textbf{A} & 6 & 3 & 1 & 4 & 2 & 5\\
\textbf{current} & \emph{i} & \emph{j} &  &  &  & \\
\end{tabular}
\end{center}
\newline
In C, this algorithm can be represented as the following; it receiving an array \emph{A}
and an total number of elements, so we can keep track where the list ends. 

\begin{listing}[H]
\begin{minted}[linenos,firstnumber=1]{c}
void insertion_sort(int *A, size_t len){
    int j, i, key;

    for (j=1; j < (int) len; j++){
        key = A[j];
        // Insert A[j] into the sorted sequence A[1.. j-1]
        i = j - 1;
        while (i >= 0 && A[i] > key){
            A[i + 1] = A[i];
            i = i - 1;
        }
        A[i + 1] = key;
    }
\end{minted}
\caption{\label{c-ss}Insertion sort algorithm in C}
\end{listing}

Line 4, we start \emph{j} at the index 1 of array A up to or equal to len which is
the total number of elements we have in array A. If \emph{len} is 0, means our array
is emtpy, so it will never enter line 4, and empty array is a sorted array. The
same works if \emph{len} is one, meaning one element in our array. Line 5, we assign \emph{key} to the
current value of A[j], we do this because otherwise we would lose the value of
A[j] at in line 9. Line 7 is simple, we get the index before \emph{j}. From line 8-10
is where the sorting begins. We check if \emph{i} is $\ge$ 0 and if A[i] is > key. If
A[i] > key is true, it means it's not sorted. So we will assign A[i + 1] = A[i]
until i > 0, shuffling the elements to the left.

\begin{table}[htb]
\caption{Insertion sort operation}
\centering
\begin{tabular}{lrrrrrrl}
\textbf{A$_{\text{1}}$} & 6 & 3 & 1 & 4 & 2 & 5 & current\\
\textbf{current$_{\text{1}}$} & \emph{i} & \emph{j} &  &  &  &  & \\
\textbf{A$_{\text{1}}$} & 3 & 6 & 1 & 4 & 2 & 5 & 1 shift\\
\textbf{current$_{\text{2}}$} &  & \emph{i} & \emph{j} &  &  &  & \\
\textbf{A$_{\text{2}}$} & 1 & 3 & 6 & 4 & 2 & 5 & 2 shifts\\
\textbf{current$_{\text{3}}$} &  &  & \emph{i} & \emph{j} &  &  & \\
\textbf{A$_{\text{3}}$} & 1 & 3 & 4 & 6 & 2 & 5 & 3 shifts\\
\textbf{current$_{\text{4}}$} &  &  &  & \emph{i} & \emph{j} &  & \\
\textbf{A$_{\text{4}}$} & 1 & 2 & 3 & 4 & 6 & 5 & 4 shifts\\
\textbf{current$_{\text{5}}$} &  &  &  &  & \emph{i} & \emph{j} & \\
\textbf{A\{5\}} & 1 & 2 & 3 & 4 & 5 & 6 & final\\
\end{tabular}
\end{table}

\subsection{Analysis}
\label{sec-2-1}
In the chapter, I am assuming you are already familiar with Algorithm analysis.

Analyzing \emph{insertion sort} algorithm can give us a good idea of when we should
use this algorithm and when we should not. Analyzing can predict the resources
the algorithm requires.

The time \emph{insertion sort} requires depends on its input; if the input is 10
elements, it would take shorter than if the input was 6*10$^{\text{3}}$. On the other hand,
insertion sort also requires a different amount of time for two input sequences of the same
size if one of the input sequences is halfway sorted.

\subsubsection{Best case}
\label{sec-2-1-1}
The best case input is an array that is already sorted. If the array is already
sorted, we have a linear running time: $\mathcal{O}(n)$.


\subsubsection{Worst case}
\label{sec-2-1-2}
The simplest worst case for \emph{insertion sort} algorithm is if the array is in
reverse order. If the array is in reversed order, the inner loop (line 8-10)
will scan and shift the \textbf{entire} "sorted" array before inserting the next
element: $\mathcal{O}(n^2)$.

\begin{center}
\begin{tabular}{lrrrrrrl}
\textbf{A$_{\text{1}}$} & 6 & 5 & 4 & 3 & 2 & 1 & current\\
\textbf{current$_{\text{1}}$} & \emph{i} & \emph{j} &  &  &  &  & \\
\textbf{A$_{\text{2}}$} & 5 & 6 & 4 & 3 & 2 & 1 & 1 shift\\
\textbf{current$_{\text{2}}$} &  & \emph{i} & \emph{j} &  &  &  & \\
\textbf{A$_{\text{3}}$} & 4 & 5 & 6 & 3 & 2 & 1 & 2 shifts\\
\textbf{current$_{\text{3}}$} &  &  & \emph{i} & \emph{j} &  &  & \\
\textbf{A$_{\text{4}}$} & 3 & 4 & 5 & 6 &  &  & 3 shifts\\
\textbf{current$_{\text{4}}$} &  &  &  & \emph{i} & \emph{j} &  & \\
\textbf{A$_{\text{5}}$} & 2 & 3 & 4 & 5 & 6 & 1 & 4 shifts\\
\textbf{current$_{\text{5}}$} &  &  &  &  & \emph{i} & \emph{j} & \\
\textbf{A$_{\text{6}}$} & 1 & 2 & 3 & 4 & 5 & 6 & 5 shifts\\
\textbf{A$_{\text{7}}$} & 1 & 2 & 3 & 4 & 5 & 6 & final\\
\end{tabular}
\end{center}

\subsubsection{Average case}
\label{sec-2-1-3}
The average sort is \textbf{quadratic}, which means that insertion sort is impractical
for sorting a large array: $\mathcal{O}(n^2)$ with a space complexity of $\mathcal{O}(1)$.
% Emacs 24.5.1 (Org mode 8.2.10)
\end{document}
